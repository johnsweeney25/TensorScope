\documentclass[11pt,a4paper]{article}
\usepackage[utf8]{inputenc}
\usepackage[T1]{fontenc}
\usepackage{amsmath,amssymb,amsfonts}
\usepackage{graphicx}
\usepackage{booktabs}
\usepackage{multirow}
\usepackage{array}
\usepackage{longtable}
\usepackage{hyperref}
\usepackage{geometry}
\usepackage{fancyhdr}
\usepackage{lastpage}
\usepackage{color}
\usepackage{xcolor}
\usepackage{listings}
\usepackage{float}
\usepackage{caption}
\usepackage{subcaption}
\usepackage{tikz}
\usepackage{pgfplots}
\pgfplotsset{compat=1.17}

% Page geometry
\geometry{
    left=25mm,
    right=25mm,
    top=30mm,
    bottom=30mm
}

% Colors
\definecolor{primarycolor}{RGB}{0,102,204}
\definecolor{secondarycolor}{RGB}{255,102,0}
\definecolor{lightgray}{RGB}{245,245,245}
\definecolor{darkgray}{RGB}{100,100,100}

% Headers and footers
\pagestyle{fancy}
\fancyhf{}
\fancyhead[L]{\small \textcolor{darkgray}{{{ title }}}}
\fancyhead[R]{\small \textcolor{darkgray}{{{ date }}}}
\fancyfoot[C]{\small Page \thepage\ of \pageref{LastPage}}
\renewcommand{\headrulewidth}{0.4pt}
\renewcommand{\footrulewidth}{0.4pt}

% Section formatting
\usepackage{titlesec}
\titleformat{\section}{\Large\bfseries\color{primarycolor}}{Section \thesection}{1em}{}
\titleformat{\subsection}{\large\bfseries\color{darkgray}}{\thesubsection}{1em}{}
\titleformat{\subsubsection}{\normalsize\bfseries}{\thesubsubsection}{1em}{}

% Code listings
\lstset{
    basicstyle=\small\ttfamily,
    backgroundcolor=\color{lightgray},
    breaklines=true,
    frame=single,
    frameround=tttt,
    numbers=left,
    numberstyle=\tiny\color{darkgray},
    keywordstyle=\color{primarycolor}\bfseries,
    commentstyle=\color{darkgray}\itshape,
    stringstyle=\color{secondarycolor}
}

% Custom commands
\newcommand{\metric}[1]{\texttt{#1}}
\newcommand{\model}[1]{\textbf{#1}}
\newcommand{\pvalue}[1]{$p = #1$}
\newcommand{\effectsize}[2]{Cohen's $d = #1$ (#2)}

% Document info
\title{
    \vspace{-2cm}
    {\Huge\bfseries\color{primarycolor} {{ title }}}\\
    \vspace{0.5cm}
    {\large Statistical Analysis of Model Performance Metrics}\\
    \vspace{0.5cm}
    \rule{\textwidth}{0.4pt}
}
\author{
    {{ author }}\\
    \small {{ affiliation }}
}
\date{{ '{' }}{{ date }}{{ '}' }}

\begin{document}

\maketitle
\thispagestyle{empty}

% Abstract/Executive Summary
\section*{Executive Summary}
\addcontentsline{toc}{section}{Executive Summary}

{{ executive_summary }}

\newpage
\tableofcontents
\newpage

% Introduction
\section{Introduction}

This report presents a comprehensive statistical analysis of model performance metrics collected from {{ analysis_description }}. The analysis encompasses {{ n_metrics }} distinct metrics evaluated across {{ n_samples }} samples.

\subsection{Objectives}

The primary objectives of this analysis are:
\begin{itemize}
    \item Characterize the distribution and behavior of performance metrics
    \item Identify significant correlations between metrics
    \item Compare performance across different model configurations
    \item Provide actionable insights for model improvement
\end{itemize}

% Methodology
\section{Methodology}

\subsection{Data Collection}

{{ data_collection }}

\subsection{Statistical Methods}

The following statistical methods were employed:

\begin{itemize}
    \item \textbf{Descriptive Statistics}: Mean, median, standard deviation, and distributional characteristics
    \item \textbf{Correlation Analysis}: Spearman rank correlation for robust correlation assessment
    \item \textbf{Hypothesis Testing}: Appropriate parametric and non-parametric tests based on data characteristics
    \item \textbf{Effect Size Calculation}: Cohen's d and Hedges' g for practical significance
    \item \textbf{Multiple Comparison Correction}: {{ correction_method }} correction to control Type I error
\end{itemize}

% Results
\section{Results}

\subsection{Descriptive Statistics}

Table~\ref{tab:descriptive} presents the descriptive statistics for key metrics.

\begin{table}[H]
\centering
\caption{Descriptive Statistics of Key Metrics}
\label{tab:descriptive}
\begin{tabular}{lrrrrr}
\toprule
\textbf{Metric} & \textbf{Mean} & \textbf{Median} & \textbf{Std} & \textbf{Min} & \textbf{Max} \\
\midrule
{{ descriptive_stats_table }}
\bottomrule
\end{tabular}
\end{table}

\subsection{Distribution Analysis}

{{ distribution_analysis }}

% Include distribution plot if available
{{ '{' }}% if distribution_plot_path %}
\begin{figure}[H]
\centering
\includegraphics[width=0.9\textwidth{{ '{' }}{{ '}' }}{{ '{' }}{{ distribution_plot_path }}{{ '}' }}
\caption{Distribution of Key Metrics}
\label{fig:distributions}
\end{figure}
{{ '{' }}% endif %}

\subsection{Correlation Analysis}

The correlation analysis revealed several significant relationships between metrics:

{{ correlation_analysis }}

% Include correlation heatmap if available
{{ '{' }}% if correlation_plot_path %}
\begin{figure}[H]
\centering
\includegraphics[width=0.8\textwidth{{ '{' }}{{ '}' }}{{ '{' }}{{ correlation_plot_path }}{{ '}' }}
\caption{Correlation Matrix of Performance Metrics}
\label{fig:correlation}
\end{figure}
{{ '{' }}% endif %}

\subsection{Group Comparisons}

{{ '{' }}% if group_comparisons %}
{{ group_comparisons }}

% Include comparison plot if available
{{ '{' }}% if comparison_plot_path %}
\begin{figure}[H]
\centering
\includegraphics[width=0.9\textwidth{{ '{' }}{{ '}' }}{{ '{' }}{{ comparison_plot_path }}{{ '}' }}
\caption{Performance Comparison Across Groups}
\label{fig:comparison}
\end{figure}
{{ '{' }}% endif %}
{{ '{' }}% endif %}

\subsection{Effect Sizes}

Table~\ref{tab:effect_sizes} presents the effect sizes for key comparisons.

\begin{table}[H]
\centering
\caption{Effect Sizes for Key Comparisons}
\label{tab:effect_sizes}
\begin{tabular}{lrrl}
\toprule
\textbf{Comparison} & \textbf{Cohen's d} & \textbf{95\% CI} & \textbf{Interpretation} \\
\midrule
{{ effect_sizes_table }}
\bottomrule
\end{tabular}
\end{table}

% Advanced Analysis
\section{Advanced Analysis}

\subsection{Principal Component Analysis}

{{ '{' }}% if pca_analysis %}
{{ pca_analysis }}

% Include PCA plot if available
{{ '{' }}% if pca_plot_path %}
\begin{figure}[H]
\centering
\includegraphics[width=0.9\textwidth{{ '{' }}{{ '}' }}{{ '{' }}{{ pca_plot_path }}{{ '}' }}
\caption{Principal Component Analysis}
\label{fig:pca}
\end{figure}
{{ '{' }}% endif %}
{{ '{' }}% endif %}

\subsection{Time Series Analysis}

{{ '{' }}% if time_series_analysis %}
{{ time_series_analysis }}

% Include time series plot if available
{{ '{' }}% if time_series_plot_path %}
\begin{figure}[H]
\centering
\includegraphics[width=0.9\textwidth{{ '{' }}{{ '}' }}{{ '{' }}{{ time_series_plot_path }}{{ '}' }}
\caption{Temporal Trends in Performance Metrics}
\label{fig:timeseries}
\end{figure}
{{ '{' }}% endif %}
{{ '{' }}% endif %}

% Discussion
\section{Discussion}

\subsection{Key Findings}

{{ key_findings }}

\subsection{Performance Insights}

{{ performance_insights }}

\subsection{Statistical Considerations}

{{ statistical_considerations }}

% Conclusions
\section{Conclusions}

{{ conclusions }}

% Recommendations
\section{Recommendations}

Based on the analysis results, we recommend:

{{ recommendations }}

% Appendices
\appendix

\section{Statistical Test Details}

\subsection{Normality Tests}

{{ normality_tests }}

\subsection{Hypothesis Test Results}

{{ hypothesis_tests }}

\section{Additional Visualizations}

{{ '{' }}% if additional_plots %}
{{ additional_plots }}
{{ '{' }}% endif %}

\section{Technical Details}

\subsection{Analysis Configuration}
\begin{itemize}
    \item \textbf{Significance Level}: $\alpha = {{ alpha }}$
    \item \textbf{Correction Method}: {{ correction_method }}
    \item \textbf{Bootstrap Iterations}: {{ bootstrap_iterations }}
    \item \textbf{Random Seed}: {{ random_seed }}
\end{itemize}

\subsection{Software and Libraries}
\begin{itemize}
    \item Python {{ python_version }}
    \item NumPy {{ numpy_version }}
    \item SciPy {{ scipy_version }}
    \item Scikit-learn {{ sklearn_version }}
    \item Statsmodels {{ statsmodels_version }}
\end{itemize}

% References
\bibliographystyle{unsrt}
\begin{thebibliography}{99}

\bibitem{tishby2000information}
N. Tishby, F. C. Pereira, and W. Bialek,
``The information bottleneck method,''
\textit{Proc. 37th Annual Allerton Conference on Communication, Control and Computing},
pp. 368--377, 2000.

\bibitem{tishby2015deep}
N. Tishby and N. Zaslavsky,
``Deep learning and the information bottleneck principle,''
\textit{IEEE Information Theory Workshop (ITW)},
pp. 1--5, 2015.

\bibitem{ji2024phase}
H. Ji, Y. Li, and S. Zhang,
``Phase transitions in neural network adaptation: Elasticity, rigidity, and catastrophic forgetting,''
\textit{arXiv preprint arXiv:2401.00000}, 2024.

\bibitem{kirkpatrick2017overcoming}
J. Kirkpatrick et al.,
``Overcoming catastrophic forgetting in neural networks,''
\textit{Proceedings of the National Academy of Sciences},
vol. 114, no. 13, pp. 3521--3526, 2017.

\bibitem{alemi2016deep}
A. A. Alemi, I. Fischer, J. V. Dillon, and K. Murphy,
``Deep variational information bottleneck,''
\textit{International Conference on Learning Representations (ICLR)}, 2017.

\bibitem{achille2018information}
A. Achille and S. Soatto,
``Information dropout: Learning optimal representations through noisy computation,''
\textit{IEEE Transactions on Pattern Analysis and Machine Intelligence},
vol. 40, no. 12, pp. 2897--2905, 2018.

\bibitem{lyapunov1992stability}
A. M. Lyapunov,
``The general problem of the stability of motion,''
\textit{International Journal of Control},
vol. 55, no. 3, pp. 531--534, 1992.

\bibitem{rissanen1978modeling}
J. Rissanen,
``Modeling by shortest data description,''
\textit{Automatica},
vol. 14, no. 5, pp. 465--471, 1978.

{{ '{' }}% if references %}
% Additional custom references
{{ references }}
{{ '{' }}% endif %}

\end{thebibliography}

\end{document}