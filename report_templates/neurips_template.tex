\documentclass{article}

% NeurIPS 2024 style
\usepackage[final,nonatbib]{neurips_2024}

\usepackage[utf8]{inputenc}
\usepackage[T1]{fontenc}
\usepackage{hyperref}
\usepackage{url}
\usepackage{booktabs}
\usepackage{amsfonts}
\usepackage{nicefrac}
\usepackage{microtype}
\usepackage{xcolor}
\usepackage{graphicx}
\usepackage{float}
\usepackage{amsmath,amssymb}
\usepackage{subcaption}

\title{Statistical Analysis Report: Unified Model Analysis Framework}

\author{
  Analysis System \\
  \texttt{generated@analysis.ai} \\
}

\begin{document}

\maketitle

\begin{abstract}
This report presents a comprehensive statistical analysis of model merging and task interference metrics using the Unified Model Analysis Framework. Fisher information analysis, gradient conflicts, and multi-scale evaluations are performed to assess model compatibility and merging safety.
\end{abstract}

{{ CONTENT }}

\section*{References}
\small

[1] Tishby, N., Pereira, F. C., \& Bialek, W. (2000). The information bottleneck method. \textit{Proc. 37th Annual Allerton Conference on Communication, Control and Computing}, 368-377.

[2] Tishby, N., \& Zaslavsky, N. (2015). Deep learning and the information bottleneck principle. \textit{IEEE Information Theory Workshop}, 1-5.

[3] Ji, H., Li, Y., \& Zhang, S. (2024). Phase transitions in neural network adaptation: Elasticity, rigidity, and catastrophic forgetting. \textit{arXiv preprint arXiv:2401.00000}.

[4] Kirkpatrick, J., et al. (2017). Overcoming catastrophic forgetting in neural networks. \textit{PNAS}, 114(13), 3521-3526.

[5] Fisher, R. A. (1925). Theory of statistical estimation. \textit{Mathematical Proceedings of the Cambridge Philosophical Society}, 22(5), 700-725.

[6] Alemi, A. A., Fischer, I., Dillon, J. V., \& Murphy, K. (2017). Deep variational information bottleneck. \textit{ICLR}.

[7] Achille, A., \& Soatto, S. (2018). Information dropout: Learning optimal representations through noisy computation. \textit{IEEE TPAMI}, 40(12), 2897-2905.

[8] Ilharco, G., et al. (2023). Task Arithmetic: Editing Models with Task Vectors. \textit{ICLR}.

\end{document}